
% Preamble
\documentclass[11pt]{article}

\newcommand*{\srcDirectory}{src}
\newcommand*{\importDirectory}{imports}
\newcommand*{\settingsDirectory}{settings}
\newcommand*{\tikzDirectory}{tikz}
\newcommand*{\importDirectDirectory}{\srcDirectory/\importDirectory}
\newcommand*{\settingsDirectDirectory}{\srcDirectory/\settingsDirectory}
\newcommand*{\tikzDirectDirectory}{\srcDirectory/\tikzDirectory}

\include{\importDirectDirectory/pl}
\usepackage{geometry}

\geometry{
    paperheight=185.625mm,
    paperwidth=297mm,
    left=20mm,
    right=20mm,
    top=20mm,
    bottom=20mm
}
\usepackage{fancyhdr}
\pagestyle{fancy}
\fancyhf{}
\fancyfoot[R]{\thepage}
\renewcommand{\headrulewidth}{0pt}
\renewcommand{\footrulewidth}{0pt}

%https://www.reddit.com/r/GradSchool/comments/cmfxjm/i_just_discovered_how_to_make_darkmode_pdfs_in/

\usepackage{xcolor}
\pagecolor[rgb]{0,0,0} %black
\color[rgb]{0.5,0.5,0.5} %grey
\include{\importDirectDirectory/math}
\usepackage{tikz}
\usetikzlibrary{shapes,fit,matrix}
\usepackage{standalone}
\tikzset{
 point/.style={circle=1mm,fill=red!60!black,scale=0.5}
}
\pgfdeclarelayer{bg}
\pgfsetlayers{bg,main} 

\begin{document}

\thispagestyle{empty}
    \null
    \vspace{5cm}
    \begin{center}
        \begin{Huge}
            \fontsize{50pt}{50pt}\selectfont{Otoczki wypukłe wielokąta prostego}
        \end{Huge}
    \end{center}
    \vspace{5cm}
    
   \begin{flushright}
   
   \Huge{Marcin Belicki} 
   \end{flushright}
        
    
    \newpage
    \begin{LARGE}
        \textbf{Otoczka wypukła} - dla zbioru punktów $Z\subset\mathbb{R}^2$ 
        jest to wielokąt wypukły $H(Z)$, taki że zawiera w swoim wnętrzu lub na krawędziach 
        wszystkie punkty $p\in Z$, a przy tym mając pole najmniejsze z możliwych.
        \begin{center}
            \includestandalone[height = 11 cm]{\tikzDirectDirectory/1}
        \end{center}    
    \end{LARGE}      
\end{document}