
% Preamble
\documentclass[11pt]{article}

\newcommand*{\srcDirectory}{src}
\newcommand*{\importDirectory}{imports}
\newcommand*{\settingsDirectory}{settings}
\newcommand*{\tikzDirectory}{tikz}
\newcommand*{\importDirectDirectory}{\srcDirectory/\importDirectory}
\newcommand*{\settingsDirectDirectory}{\srcDirectory/\settingsDirectory}
\newcommand*{\tikzDirectDirectory}{\srcDirectory/\tikzDirectory}

\include{\importDirectDirectory/pl}
\usepackage{geometry}

\geometry{
    paperheight=185.625mm,
    paperwidth=297mm,
    left=20mm,
    right=20mm,
    top=20mm,
    bottom=20mm
}
\usepackage{fancyhdr}
\pagestyle{fancy}
\fancyhf{}
\fancyfoot[R]{\thepage}
\renewcommand{\headrulewidth}{0pt}
\renewcommand{\footrulewidth}{0pt}

%https://www.reddit.com/r/GradSchool/comments/cmfxjm/i_just_discovered_how_to_make_darkmode_pdfs_in/

\usepackage{xcolor}
\pagecolor[rgb]{0,0,0} %black
\color[rgb]{0.5,0.5,0.5} %grey
\include{\importDirectDirectory/math}
\usepackage{tikz}
\usetikzlibrary{shapes,fit,matrix}
\usepackage{standalone}
\tikzset{
 point/.style={circle=1mm,fill=red!60!black,scale=0.5}
}
\pgfdeclarelayer{bg}
\pgfsetlayers{bg,main} 

%https://tex.stackexchange.com/questions/58903/how-to-draw-star-in-tikz-background
\newcommand{\tstar}[6]{% inner radius, outer radius, tips, rot angle, options, name
    \pgfmathsetmacro{\starangle}{360/#3}
    \draw[#5] (#4:#1) coordinate (#6-0-0)
    \foreach \x in {1,...,#3} { 
        -- (#4+\x*\starangle-\starangle/2:#2) coordinate (#6-\x-0) 
        -- (#4+\x*\starangle:#1)              coordinate (#6-\x-1)
    }
    -- cycle;
}

%https://tex.stackexchange.com/questions/58903/how-to-draw-star-in-tikz-background
\newcommand{\ngram}[5]{% outer radius, tips, rot angle, options
    \pgfmathsetmacro{\starangle}{360/#2}
    \pgfmathsetmacro{\innerradius}{#1*sin(90-\starangle)/sin(90+\starangle/2)}
    \tstar{\innerradius}{#1}{#2}{#3}{#4}{#5}
}



\begin{document}

\thispagestyle{empty}
    \null
    \vspace{5cm}
    \begin{center}
        \begin{Huge}
            \fontsize{50pt}{50pt}\selectfont{Otoczki wypukłe wielokąta prostego}
        \end{Huge}
    \end{center}
    \vspace{5cm}
    
   \begin{flushright}
   
   \Huge{Marcin Belicki} 
   \end{flushright}
        
    
    \newpage
    \begin{center}
        \begin{Huge}
            \fontsize{46pt}{46pt}\selectfont{Definicja}
        \end{Huge}   
    \end{center}
    \begin{LARGE}
        \textbf{Otoczka wypukła} - dla zbioru punktów $Z\subset\mathbb{R}^2$ 
        jest to wielokąt wypukły $H(Z)$, taki że zawiera w swoim wnętrzu lub na krawędziach 
        wszystkie punkty $p\in Z$, a przy tym mając pole najmniejsze z możliwych.
        \begin{center}
            \includestandalone[height = 10 cm]{\tikzDirectDirectory/1}
        \end{center}    
    \end{LARGE}   
    
    \newpage
    \begin{center}
    \begin{Huge}
        \fontsize{46pt}{46pt}\selectfont{Zastosowania}
    \end{Huge}   
\end{center}
\begin{LARGE}
    Właściwości otoczki wypukłej można wykorzystać w grafice komputerowej. 
    Przykładem tego może być wykrywanie przez silnik graficzny kolizji między obiektami.
    Jeśli wiemy, że obiekt czeka kolizja z płaską powierzchnią możemy przyjąć,
    że kolizja nastąpi wtedy i tylko wtedy gdy nastąpi kolizja otoczki wypukłej tego obiektu z powierzchnią.
\end{LARGE}
\begin{center}
    \includestandalone{\tikzDirectDirectory/2_0}
\end{center} 

    \newpage
    \begin{center}
    \begin{Huge}
        \fontsize{46pt}{46pt}\selectfont{Zastosowania}
    \end{Huge}   
\end{center}
\begin{LARGE}
    Właściwości otoczki wypukłej można wykorzystać w grafice komputerowej. 
    Przykładem tego może być wykrywanie przez silnik graficzny kolizji między obiektami.
    Jeśli wiemy, że obiekt czeka kolizja z płaską powierzchnią możemy przyjąć,
    że kolizja nastąpi wtedy i tylko wtedy gdy nastąpi kolizja otoczki wypukłej tego obiektu z powierzchnią.
\end{LARGE}
\begin{center}
    \includestandalone{\tikzDirectDirectory/2_1}
\end{center}

    \newpage
    \begin{center}
    \begin{Huge}
        \fontsize{46pt}{46pt}\selectfont{Zastosowania}
    \end{Huge}   
\end{center}
\begin{LARGE}
    Właściwości otoczki wypukłej można wykorzystać w grafice komputerowej. 
    Przykładem tego może być wykrywanie przez silnik graficzny kolizji między obiektami.
    Jeśli wiemy, że obiekt czeka kolizja z płaską powierzchnią możemy przyjąć,
    że kolizja nastąpi wtedy i tylko wtedy gdy nastąpi kolizja otoczki wypukłej tego obiektu z powierzchnią.
\end{LARGE}
\begin{center}
    \includestandalone{\tikzDirectDirectory/2_2}
\end{center}

    \newpage
    \begin{center}
    \begin{Huge}
        \fontsize{46pt}{46pt}\selectfont{Zastosowania}
    \end{Huge}   
\end{center}
\begin{LARGE}
    Właściwości otoczki wypukłej można wykorzystać w grafice komputerowej. 
    Przykładem tego może być wykrywanie przez silnik graficzny kolizji między obiektami.
    Jeśli wiemy, że obiekt czeka kolizja z płaską powierzchnią możemy przyjąć,
    że kolizja nastąpi wtedy i tylko wtedy gdy nastąpi kolizja otoczki wypukłej tego obiektu z powierzchnią.
\end{LARGE}
\vspace*{\fill}
\begin{center}
    \includestandalone{\tikzDirectDirectory/2_3}
\end{center}

    \newpage
    \begin{center}
    \begin{Huge}
        \fontsize{46pt}{46pt}\selectfont{Zastosowania}
    \end{Huge}   
\end{center}
\begin{LARGE}
    Dzięki reprezentacji obiektu przez jego otoczkę wypukłą uzyskujemy mniej skomplikowany wielokąt,
    przez co łatwiej jest przeprowadzić nam obliczenia potrzebne do wykrycia kolizji.
\end{LARGE}
\vspace*{\fill}
\begin{center}
    \includestandalone{\tikzDirectDirectory/2_3}
\end{center}
    

\end{document}