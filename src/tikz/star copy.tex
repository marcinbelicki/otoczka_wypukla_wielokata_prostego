\begin{center}
    \begin{Huge}
        \fontsize{46pt}{46pt}\selectfont{Zastosowania}
    \end{Huge}   
\end{center}
\begin{LARGE}
    Właściwości otoczki wypukłej można wykorzystać w grafice komputerowej. 
    Przykładem tego może być wykrywanie przez silnik graficzny kolizji między obiektami.
    Jeśli wiemy, że obiekt czeka kolizja z płaską powierzchnią możemy przyjąć,
    że kolizja nastąpi wtedy i tylko wtedy gdy nastąpi kolizja otoczki wypukłej tego obiektu z powierzchnią.
\end{LARGE}